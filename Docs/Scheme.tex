\documentclass[a4paper]{article}

\usepackage[margin = 2.0cm]{geometry}
\usepackage{graphicx}
\usepackage{color}

\begin{document}

\newcommand{\U}{\mathbf{U}}
\newcommand{\E}{\mathbf{E}}
\newcommand{\F}{\mathbf{F}}
\newcommand{\G}{\mathbf{G}}
\newcommand{\HH}{\mathbf{H}}
\newcommand{\I}{\mathbf{I}}
\newcommand{\w}{\mathbf{w}}
\newcommand{\R}{\mathbf{R}}
\newcommand{\Sn}{\mathbf{S}}

\section{Turbulence Models}

\begin{eqnarray}
\frac{D \rho k}{Dt}
&=&
\frac{\partial}{\partial x_j}
\left[
\left(
\mu + \sigma_k \mu_T
\right)
\frac{\partial k}{\partial x_j}
\right]
-
\beta_k \rho k \omega
+
\tau_{ij}\frac{\partial u_i}{\partial x_j}
\label{Eq:KTransport}
\\
\frac{D \rho \omega}{Dt}
&=&
\frac{\partial}{\partial x_j}
\left[
\left(
\mu + \sigma_\omega \mu_T
\right)
\frac{\partial \omega}{\partial x_j}
\right]
-
\beta_\omega \rho \omega^2
+
\alpha_\omega
\frac{\omega}{k}
\tau_{ij}\frac{\partial u_i}{\partial x_j}
\label{Eq:OmegaTransport}
\end{eqnarray}

\[
k = \frac{u_i u_i}{2} = \frac{[L^2]}{[T^2]}
\]

\[
\tau_{ij} = \rho u_i u_j = 2 \mu_T S_{ij} = \frac{[M]}{[L][T^2]}
\]

\[
S_{ij} = \frac{\partial \tilde{u}_i}{\partial x_j} + \frac{\partial \tilde{u}_j}{\partial x_i} = [T^{-1}]
\]

\[
\rho = \frac{[M]}{[L^3]}
\]

If we choose dissipation per unit turbulence kinetic energy, $\omega$, as the second parameter, which has the units
\[
\omega = [T^{-1}]
\]
then the turbulence viscosity coefficient can be expressed as
\[
\mu_T = \frac{[M]}{[L][T]}
= \frac{\rho k}{\omega}
.
\]

\subsection{Nondimensionalization}

We employ the following nondimensionalization.
%
%\begin{eqnarray}
%k^* &=& \frac{k}{c_\infty^2} \label{Eq:KNondim} \\
%\omega^* &=& \frac{\omega L}{c_\infty} \label{Eq:OmegaNondim}
%\end{eqnarray}
%
%Using Eqs.~\ref{Eq:KNondim} and \ref{Eq:OmegaNondim}, Eqs.~\ref{Eq:KTransport} and \ref{Eq:OmegaTransport} are nondimensionalized as
%
%\begin{equation}
%\rho^* \frac{\partial k^*}{\partial t^*} + \rho^* U^*_j \frac{\partial k^*}{\partial x^*_j}
%=
%\frac{1}{Re} \tau^*_{ij} \frac{\partial U^*_i}{\partial x^*_j}
%- \beta_k \rho^* k^* \omega^*
%+ \frac{1}{Re} \frac{\partial}{\partial x^*_j} \left[
%	\left( \mu^* + \sigma_k \mu_T^* \right) \frac{\partial k^*}{\partial x^*_j}
%	\right]
%\end{equation}
%
%\begin{equation}
%\rho^* \frac{\partial \omega^*}{\partial t^*} + \rho^* U^*_j \frac{\partial \omega^*}{\partial x^*_j}
%=
%\frac{1}{Re} \frac{\alpha_\omega \omega^*}{k^*} \tau^*_{ij} \frac{\partial U^*_i}{\partial x^*_j}
%- \beta_\omega \rho^* {\omega^*}^2
%+ \frac{1}{Re} \frac{\partial}{\partial x^*_j} \left[
%	\left( \mu^* + \sigma_\omega \mu_T^* \right) \frac{\partial \omega^*}{\partial x^*_j}
%	\right]
%\end{equation}
%
%Note that the above nondimensionalization convention is different from that used in various NASA codes such as CFL3D and OVERFLOW which is given below as reference.
\begin{eqnarray}
k^* &=& \frac{k}{c_\infty^2} \label{Eq:KNondim} \\
\omega^* &=& \frac{\omega \mu_\infty}{\rho_\infty c_\infty^2} \label{Eq:OmegaNondim}
\end{eqnarray}
%
from which we express the nondimensional turbulent viscosity, $\mu_T^*$, as follows.
%
\begin{equation}
\mu_T^* = \frac{\mu_T}{\mu_\infty}
	= \frac{\rho k}{\omega \mu_\infty}
	= \frac{\rho^* \rho_\infty k^* c_\infty^2}{\omega^* \left(\frac{\rho_\infty c_\infty^2}{\mu_\infty}\right) \mu_\infty}
	= \left(\frac{\rho_\infty c_\infty^2}{\rho_\infty c_\infty^2}\right) \frac{\rho^* k^*}{\omega^*}
	= \frac{\rho^* k^*}{\omega^*}
\end{equation}
%
%using the CFL3D convention, the $k$-$\omega$ turbulence model is written in nondimensional form as
Using Eqs.~\ref{Eq:KNondim} and \ref{Eq:OmegaNondim}, Eqs.~\ref{Eq:KTransport} and \ref{Eq:OmegaTransport} are nondimensionalized as
%
\begin{equation}
\rho^* \frac{\partial k^*}{\partial t^*} + \rho^* U^*_j \frac{\partial k^*}{\partial x^*_j}
=
\frac{1}{Re} \tau^*_{ij} \frac{\partial U^*_i}{\partial x^*_j}
- Re \beta_k \rho^* k^* \omega^*
+ \frac{1}{Re} \frac{\partial}{\partial x^*_j} \left[
	\left( \mu^* + \sigma_k \mu_T^* \right) \frac{\partial k^*}{\partial x^*_j}
	\right]
\end{equation}
%
\begin{equation}
\rho^* \frac{\partial \omega^*}{\partial t^*} + \rho^* U^*_j \frac{\partial \omega^*}{\partial x^*_j}
=
\frac{1}{Re} \frac{\alpha_\omega \omega^*}{k^*} \tau^*_{ij} \frac{\partial U^*_i}{\partial x^*_j}
- Re \beta_\omega \rho^* {\omega^*}^2
+ \frac{1}{Re} \frac{\partial}{\partial x^*_j} \left[
	\left( \mu^* + \sigma_\omega \mu_T^* \right) \frac{\partial \omega^*}{\partial x^*_j}
	\right]
\end{equation}



\section{Governing Equation}
\begin{equation}
\frac{\partial\U}{\partial t} + \R\left(\U\right) = 0
\label{Eq:GovEq}
\end{equation}

% $Id: GoverningEquations.tex 272 2013-02-21 08:09:44Z kato $

\section{Governing Equations}

\subsection{Nondimensionalization}

The governing equations are nondimensionalized with the criteria that the resulting set of equations be easily applicable to both internal and external flows and that they render the equation of state
\begin{eqnarray*}
p &=& \rho R T \\
  &=& \rho e \left(\gamma - 1\right)
\end{eqnarray*}
as simple as possible from the viewpoint of modular implementation. For these reasons, the choice has been made to use the following nondimensionalization.
\begin{eqnarray*}
x^* &=& \frac{x}{L} \\
t^* &=& \frac{t c_{\infty}}{L} \\
\mathbf{V}^* &=& \frac{V}{c_{\infty}} \\
\rho^* &=& \frac{\rho}{\rho_\infty} \\
p^* &=& \frac{p}{\rho_\infty c_\infty^2} \left( = \frac{p}{\gamma_\infty p_\infty} \right) \\
T^* &=& \frac{T R_\infty}{c_\infty^2} \left( = \frac{T}{\gamma_\infty T_\infty} \right) \\
e^* &=& \frac{e}{c_\infty^2} \\
R^* &=& \frac{R}{R_\infty} \\
\mu^* &=& \frac{\mu}{\mu_\infty}
\end{eqnarray*}
where the quantities with asterisk denote nondimentionalized variables. With this choice of nondimensionalization the following useful relations follow which are taken advantaged of throughout the code.
\begin{eqnarray*}
{c^*}^2 &=& \gamma T^* \\
p^* &=& \rho^* T^*
\end{eqnarray*}

%In a similar fashion, the solution variables involved in turbulence modeling can be nondimensionalized as follows. The $k-\omega$ two-equation model solves turbulence kinetic energy, $k$, and specific dissipation rate, $\omega$, which are nondimensionalized as
%\begin{eqnarray*}
%k^* &=& \frac{k}{c_\infty^2} \\
%\omega^* &=& \frac{\omega L}{c_\infty} \\
%\mu_T^* &=& \frac{\mu_T}{\mu_\infty}
%	= \frac{\rho k}{\omega \mu_\infty}
%	= \frac{\rho^* \rho_\infty k^* c_\infty^2}{\omega^* \frac{c_\infty}{L} \mu_\infty}
%	= \left(\frac{\rho_\infty c_\infty L}{\mu_\infty}\right) \frac{\rho^* k^*}{\omega^*}
%	= Re \frac{\rho^* k^*}{\omega^*}
%\end{eqnarray*}


\section{Temporal Integration}

\subsection{Steady Simulation}
We seek a solution to the governing equation
\begin{equation}
\R\left(\U\right) = 0
\end{equation}
which can be iteratively solved using Newton's method
\begin{equation}
\left(\frac{\partial \R}{\partial\U}\right)^k
\Delta \U^k
=
\R\left(\U^k\right)
\end{equation}
where $\Delta\U^k = \U^{k + 1} - \U^k$.

\subsection{Unsteady Simulation}

Approximating the time derivative term in Eq.~\ref{Eq:GovEq} with the first-order backward finite difference we obtain the so-called first-order backward Euler scheme, which is given as follows.
\begin{equation}
\frac{\U^{n+1} - \U^n}{\Delta t} + \R\left(\U^{n+1}\right) = 0
\label{Eq:FirstOrderBackwardEuler}
\end{equation}
For second-order temporal accuracy, three-point backward differencing of the time derivative yields
\begin{equation}
\frac{3\U^{n+1} - 4\U^n + \U^{n-1}}{2\Delta t} + \R\left(\U^{n+1}\right) = 0.
\label{Eq:SecondOrderBackwardEuler}
\end{equation}

\subsubsection{Dual Time Stepping}

We introduce an artificial time, $\tau$, along which integration is carried out. By separating the physical time, $t$, from numerical integration any implicit method can be employed (in the artificial time) to advance the solution in the physical time regardless of the physical time step.
\begin{equation}
\frac{\partial\U}{\partial t} + \frac{\partial\U}{\partial \tau} + \R\left(\U\right) = 0
\end{equation}
Denoting the physical time index with $n$ and the artificial time index with $k$, we have
\[
\frac{3\U^{n+1} - 4\U^n + \U^{n-1}}{2\Delta t} + \frac{\U^{k+1} - \U^k}{\Delta \tau} +\R\left(\U^{n+1}\right) = 0.
\]
As $k \rightarrow \infty$, we want to drive $\U^{k + 1} \rightarrow \U^{n + 1}$,
\begin{eqnarray}
\frac{3\U^{k+1} - 4\U^n + \U^{n-1}}{2\Delta t} + \frac{\U^{k+1} - \U^k}{\Delta \tau} +\R\left(\U^{k+1}\right) &=& 0 \nonumber \\
\frac{3\U^{k+1} - 3\U^k + 3\U^k - 4\U^n + \U^{n-1}}{2\Delta t} + \frac{\U^{k+1} - \U^k}{\Delta \tau} +\R\left(\U^{k+1}\right) &=& 0 \nonumber \\
\left(\frac{3}{2\Delta t} + \frac{1}{\Delta \tau}\right) \Delta^k\U + \frac{3\U^k - 4\U^n + \U^{n-1}}{2\Delta t} +\R\left(\U^{k+1}\right) &=& 0 \nonumber \\
\left(\frac{3}{2\Delta t} + \frac{1}{\Delta \tau}\right) \Delta^k\U + \frac{\partial\R}{\partial\U}\left(\U^k\right) \Delta^k\U &=& -\R\left(\U^k\right) - \frac{3\U^k - 4\U^n + \U^{n-1}}{2\Delta t}
\end{eqnarray}
where $\Delta^k\U = \U^{k+1} - \U^k$.

\subsubsection{Newton's Method}

In order to pose Eq.~\ref{Eq:FirstOrderBackwardEuler} in a form suitable for Neton's method, we define the left hand side as the new residual denoted by $\R^*$ and replace $\U^{n+1}$ with $\U$.
\begin{equation}
\R^*\left(\U\right) =
\frac{\U - \U^n}{\Delta t} + \R\left(\U\right) = 0
\label{Eq:FirstOrderBackwardEulerResidual}
\end{equation}
The Newton's method will be employed to iteratively solve for $\U$ which drives the residual defined by Eq.~\ref{Eq:FirstOrderBackwardEulerResidual} to zero, or
\begin{equation}
\left(\frac{\partial \R^*}{\partial\U}\right)^k
\Delta \U^k
=
-\R^*\left(\U^k\right).
\end{equation}

Similarly second-order backward discretization of the time derivative term yields
\begin{equation}
\frac{3\U^{n+1} - 4\U^n + \U^{n - 1}}{2\Delta t} + \R\left(\U^{n+1}\right) = 0
\end{equation}
for which the corresponding modified residual is given by
\begin{equation}
\R^*\left(\U\right) =
\frac{3\U - 4\U^n + \U^{n - 1}}{2\Delta t} + \R\left(\U\right) = 0 .
\end{equation}

\subsection{LUSGS-Newton}

\begin{equation}
\R\left(\U\right)
=
\sum_{j \in N_{i}} \HH_{ij} \cdot \Sn_{ij}
=
\sum_{j \in N^-_{i}} \HH_{ij} \cdot \Sn_{ij}
+
\sum_{j \in N^+_{i}} \HH_{ij} \cdot \Sn_{ij}
\end{equation}
In practice we do not evaluate the numerical flux dyad $\HH$ and the surface normal $\Sn$ separately, rather evaluate the term at once. Thus we define its shorthand as
\[
\tilde{\HH}_{ij} \equiv \HH_{ij} \cdot \Sn_{ij}
\]

\begin{equation}
\R\left(\U\right)
=
\sum_{j \in N_{i}} \tilde{\HH}_{ij}
=
\sum_{j \in N^-_{i}} \tilde{\HH}_{ij}
+
\sum_{j \in N^+_{i}} \tilde{\HH}_{ij}
\end{equation}

\begin{eqnarray}
\frac{\partial\R}{\partial\U} \Delta\U
&=&
\left(
\sum_{j \in N^-_{i}} \frac{\partial\tilde{\HH}_{ij}}{\partial\U}
+
\sum_{j \in N^+_{i}} \frac{\partial\tilde{\HH}_{ij}}{\partial\U}
\right)
\Delta\U
\end{eqnarray}

\[
\frac{\partial\tilde{\HH}_{ij}}{\partial\U}
=
\frac{\partial\tilde{\HH}_{ij}}{\partial\U_i}
+
\frac{\partial\tilde{\HH}_{ij}}{\partial\U_j}
\]

\[
\tilde{\HH}_{ij}
=
\frac{1}{2}\left(\HH_i + \HH_j\right) \cdot \Sn_{ij}
-
\frac{1}{2} \tilde{\lambda}_{ij} \left(\U_j - \U_i\right)
\]

\begin{eqnarray*}
\frac{\partial\R}{\partial\U} \Delta\U
&=&
\left(
\sum_{j \in N_i}
\frac{\partial\tilde{\HH}_{ij}}{\partial\U}
\right) \Delta\U
\\
&=&
\left[
\sum_{j \in N_i}
\left(
\frac{\partial\tilde{\HH}_{ij}}{\partial\U_i}
+
\frac{\partial\tilde{\HH}_{ij}}{\partial\U_j}
\right)
\right] \Delta\U
\\
&=&
\sum_{j \in N_i}
\left(
\frac{1}{2}\frac{\partial\HH_i}{\partial\U_i} \cdot \Sn_{ij} \Delta\U_i
+
\frac{1}{2}\frac{\partial\HH_j}{\partial\U_j} \cdot \Sn_{ij} \Delta\U_j
+
\frac{1}{2}\tilde{\lambda}_{ij} \Delta\U_i
-
\frac{1}{2}\tilde{\lambda}_{ij} \Delta\U_j
\right)
\end{eqnarray*}

\subsection{Matrix-free Method}
\begin{equation}
\left(\frac{\partial R^*}{\partial\U}\right)^k \w
\approx
\frac{R^*\left(\U^k + \epsilon\w\right) - R^*\left(\U^k\right)}{\epsilon}
\end{equation}
\begin{equation}
\end{equation}

\section{Newton-Krylov Scheme}

Given a generic residual
\[
\mathbf{R}(\mathbf{U}) = \mathbf{R}_F(\mathbf{U}) + \mathbf{Q}(\mathbf{U})
\]
where $\mathbf{R}_F$ is the residual due to inviscid and viscous fluxes and $\mathbf{Q}$ is the source term.

We seek a solution to the equation
\[
\mathbf{R}(\mathbf{U}) = 0
\]
by using Newton iteration
\begin{equation}
\left(\frac{\partial\mathbf{R}(\mathbf{U})}{\partial\mathbf{U}}\right)^n \Delta\mathbf{U}
=
-\mathbf{R}(\mathbf{U}^n)
\label{Eq:NewtonFormula}
\end{equation}

The Generalized Minimum Residual (GMRES) Method iteratively solves an equation of the form
\[
A\mathbf{x} = \mathbf{b}
\]
by seeking a solution in the Krylov subspace
\[
K_n = span\{ \mathbf{b}, A\mathbf{b}, A\mathbf{b}^2, A\mathbf{b}^3, \ldots, A\mathbf{b}^n \}
\]

Referring to Eq.~\ref{Eq:NewtonFormula}, wee see that
\[
A \equiv \left(\frac{\partial\mathbf{R}(\mathbf{U})}{\partial\mathbf{U}}\right)^n
\]
The Krylov subspace is constructed by successively evaluating the following matrix-vector product
\[
A\mathbf{v} = 
\left(\frac{\partial\mathbf{R}(\mathbf{U})}{\partial\mathbf{U}}\right)^n \mathbf{v}
\]
which can be further simplified by the following approximation
\[
\left(\frac{\partial\mathbf{R}(\mathbf{U})}{\partial\mathbf{U}}\right)^n \mathbf{v}
\approx
\frac{
\mathbf{R}(\mathbf{U}^n + \epsilon v) - \mathbf{R}(\mathbf{U}^n)
}
{
\epsilon
}
\]

\section{Miscellany}

\end{document}

